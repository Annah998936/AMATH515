\documentclass[11pt]{amsart}
\usepackage{geometry}                % See geometry.pdf to learn the layout options. There are lots.
\geometry{letterpaper}                   % ... or a4paper or a5paper or ... 
%\geometry{landscape}                % Activate for for rotated page geometry
%\usepackage[parfill]{parskip}    % Activate to begin paragraphs with an empty line rather than an indent
\usepackage{graphicx}
\usepackage{amssymb}
\usepackage[all]{xy}
\usepackage{epstopdf}
\usepackage{hyperref}

\DeclareGraphicsRule{.tif}{png}{.png}{`convert #1 `dirname #1`/`basename #1 .tif`.png}


% these packages make it easy to include figures in the text. 
\usepackage{float}
\restylefloat{figure}

\newcommand{\cX}{\mathcal{X}}
\newcommand{\cC}{\mathcal{C}}
\newcommand{\cF}{\mathcal{F}}




\begin{document}
{\Large Name:}  \\
\begin{center}
\Large AMATH 515 \hskip 2in Homework Set 0\\
{\bf Due:  Wednesday, January 11th, by 11 pm}. 
\end{center}

\bigskip

%\item Read section 1.5 of the course notes (LivingText) and show the following: 
%\begin{enumerate}
%\item First part of Corollary 1.14. Given any $\beta$-smooth function $f: U \rightarrow \mathbb{R}$, for any points $x,y \in U$, the inequality 
%\[
%\left| f(y) - f(x) - \nabla f(x)^T(y-x)\right| \leq \frac{\beta}{2}\|y-x\|^2
%\]
%holds. 
%
%\item In class we saw that the operator norm on $\nabla^2 f(x)$ gives a $\beta$. Show the opposite direction, i.e. that if $f$ is $\beta$-smooth 
%and twice-continuously differentiable, then $\|\nabla^2 f\| \leq \beta$.
%
%\end{enumerate}
%
%\bigskip\bigskip

The goal of this homework is to make sure you are comfortable with all prerequisites for this class, to set-up Python and Jupyter Notebook, and to try submitting your work to Gradescope Autograder. 
The theoretical portion of the homework will be graded based on completeness, and is intended as a primer on calculus and linear algebra. 


\section{Theory}
\begin{enumerate}
	\item Submit your write-up to Gradescope. Look for the assignment "Homework 0 -- theory".

\item {\bf Calculus primer}. For a function $f:\mathbb{R}^n \rightarrow \mathbb{R}$, we define the 
{\it gradient} to be the vector of partial derivatives: 
\[
\nabla f(x) = \begin{bmatrix} \frac{\partial f}{\partial x_1} \\ \vdots \\ \frac{\partial f}{\partial x_n}\end{bmatrix}
\]
and the {\it Hessian} to be the matrix of second partial derivatives: 
\[
\nabla^2 f(x) = 
\begin{bmatrix} \frac{\partial^2 f}{\partial x_1 \partial x_1}  & \dots & \frac{\partial^2 f}{\partial x_1 \partial x_n}
\\ \vdots 
\\ \frac{\partial^2 f}{\partial x_n \partial x_1} & \dots & \frac{\partial^2 f}{\partial x_n \partial x_n}\end{bmatrix}
\]
Compute the gradients and hessians of the following functions, with $x \in \mathbb{R}^4$ in all three examples. 
\begin{enumerate}
\item $f(x) = \sin(x_1 + x_2 + x_3 + x_4)$ \\
\item $f(x) = \|x\|^2 = x_1^2 + x_2^2 + x_3^2 + x_4^2$\\
\item $f(x) = \ln(x_1x_2x_3x_4)$.  
\end{enumerate}

\newpage

\item {\bf Linear algebra primer.}
\begin{enumerate}
\item What are the eigenvalues of the following matrix: 
\[
\begin{bmatrix}
1 & 0 & 0 & 0 \\
\pi & 2 & 0 & 0 \\
64 & -15 & 3 & 0 \\ 
321 & 0 & 0 & 5 
\end{bmatrix}
\]

\item Write down bases for the range and nullspace of the following matrix, written as the outer product of two vectors: 
\[
A = \begin{bmatrix}
1 \\ 0 \\ 1
\end{bmatrix} 
\begin{bmatrix}
1 & 1 & 1
\end{bmatrix} 
\]

\item Let $A$ be a $10 \times 5$ matrix, and $b$ a vector in $\mathbb{R}^{10}$.  The notation $A^T$ denotes the {\it transpose} of $A$, where the columns of $A$ are rows of $A^T$.
\begin{itemize}
\item What is the size of $A^TA$? What is the size of $A^Tb$?
\item How many solutions might there be to the system $Ax = b$? 
\item How many solutions might there be to the system $A^TA x = A^T b$?
\item Suppose the columns of $A$ are linearly independent. How many solutions might there be to the system $Ax=b$? To the system $A^TAx = A^Tb$?
\end{itemize}




\end{enumerate} 



\end{enumerate}

\bigskip \bigskip


\section{Practice}
\begin{enumerate}
	\item Install Anaconda3 distribution. Instruction:\\ \href{https://www.anaconda.com/products/individual}{https://www.anaconda.com/products/individual}
		\begin{itemize}
			\item 		If you've never used Python before -- here is an excellent Python introduction: \\  \href{https://www.learnpython.org}{https://www.learnpython.org}
			\item If you have experience with scientific computing in MATLAB, but you've never tried Python, here is a useful migration guide: \\\href{https://www.enthought.com/white-paper-matlab-to-python-a-migration-guide/}{https://www.enthought.com/white-paper-matlab-to-python-a-migration-guide/}
		\end{itemize}
	\item Download "Homework0.ipynb" from Canvas, open it as a Jupyter Notebook, and complete all the tasks there.
		\begin{itemize}
			\item If you've never used Jupyter Notebooks then take a look at this tutorial:
		\href{https://www.dataquest.io/blog/jupyter-notebook-tutorial/}{https://www.dataquest.io/blog/jupyter-notebook-tutorial/}
		\end{itemize}

	\item Submit your Jupyter Notebook to Gradescope. Look for the assignment "Homework 0 -- practice". There is no limit for the number of attempts for the coding part this time. 
\end{enumerate}

\end{document}  
